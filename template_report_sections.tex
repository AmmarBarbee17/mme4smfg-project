%==============================================================================
% TEMPLATE: DETAILED REPORT SECTIONS
%==============================================================================
%
% PURPOSE:
%   This file contains the detailed content for the REPORT mode.
%   It includes comprehensive sections with full explanations, derivations,
%   and extensive discussion that would be too detailed for a presentation.
%
% IMPORTANT FEATURES:
%   - Each section has BOTH \label{} and \hypertarget{} commands
%   - \label{} is used for internal cross-references within the report
%   - \hypertarget{} creates named anchors for hyperlinks from the presentation
%
% HYPERLINK TARGETS:
%   The \hypertarget{name}{} commands create "anchors" that can be referenced
%   from the presentation slides using URLs like:
%   https://your-site.github.io/report.pdf#intro
%                                                ^^^^^ matches \hypertarget{intro}{}
%
% CUSTOMIZATION:
%   Replace all placeholder text with your actual research content.
%   Keep the \label{} and \hypertarget{} commands - they enable the hyperlink system.
%   Add more sections/subsections as needed following the same pattern.
%
%==============================================================================

%------------------------------------------------------------------------------
% SECTION 1: INTRODUCTION
%------------------------------------------------------------------------------
% This section appears on approximately page 2 of the report (after title/TOC)
\section{Introduction}
\label{sec:introduction}          % For internal references: \ref{sec:introduction}
\hypertarget{intro}{}              % For external links: report.pdf#intro

% ← REPLACE THIS PARAGRAPH with your introduction
This is the detailed introduction section of the report. Here you can provide 
comprehensive background, context, and motivation for your work.

% EXAMPLE: What makes a good introduction
In academic and technical writing, it's important to:
\begin{itemize}
    \item Establish the problem context - Why does this matter?
    \item Review relevant literature - What has been done before?
    \item State research objectives clearly - What are you trying to achieve?
    \item Outline the document structure - How is the rest organized?
\end{itemize}

%..............................................................................
% SUBSECTION: Background
%..............................................................................
\subsection{Background}
\label{subsec:background}
\hypertarget{background}{}         % External link: report.pdf#background

% ← REPLACE with your background information
Detailed background information goes here. This section can include:
\begin{itemize}
    \item Extensive literature review
    \item Historical context of the problem
    \item Foundational concepts necessary for understanding your work
    \item Previous approaches and their limitations
\end{itemize}

%..............................................................................
% SUBSECTION: Objectives
%..............................................................................
\subsection{Objectives}
\label{subsec:objectives}
\hypertarget{objectives}{}         % External link: report.pdf#objectives

% ← REPLACE with your specific objectives
State your specific objectives and goals:
\begin{enumerate}
    \item First objective with detailed explanation and rationale
    \item Second objective with context and expected outcomes
    \item Third objective with methodology and success criteria
\end{enumerate}

%------------------------------------------------------------------------------
% SECTION 2: METHODOLOGY
%------------------------------------------------------------------------------
\section{Methodology}
\label{sec:methodology}
\hypertarget{methodology}{}        % External link: report.pdf#methodology

% ← REPLACE with your methodology description
This section provides detailed methodology that would be too extensive for 
a presentation slide. Include step-by-step procedures, equipment specifications,
and justification for your chosen methods.

%..............................................................................
% SUBSECTION: Experimental Setup
%..............................................................................
\subsection{Experimental Setup}
\label{subsec:setup}
\hypertarget{setup}{}              % External link: report.pdf#setup

% ← REPLACE with your experimental/analytical setup
Describe your experimental or analytical setup in detail. Include:
\begin{itemize}
    \item Equipment specifications (model numbers, calibration details)
    \item Software tools and versions used in analysis
    \item Data collection procedures and sampling protocols
    \item Quality control measures and validation procedures
    \item Environmental conditions or constraints
\end{itemize}

% EXAMPLE: You might include a detailed equipment list:
% \begin{table}[h]
%     \centering
%     \begin{tabular}{ll}
%         \hline
%         Equipment & Specifications \\
%         \hline
%         Microscope & Zeiss Model XYZ, 1000x magnification \\
%         Camera & Canon EOS R5, 45MP sensor \\
%         \hline
%     \end{tabular}
%     \caption{Equipment specifications}
% \end{table}

%..............................................................................
% SUBSECTION: Data Analysis
%..............................................................................
\subsection{Data Analysis}
\label{subsec:analysis}
\hypertarget{analysis}{}           % External link: report.pdf#analysis

% ← REPLACE with your data analysis methods
Explain your data analysis approach in detail:

% EXAMPLE: Mathematical model
\begin{equation}
\label{eq:main}
y = f(x, \theta) = \sum_{i=1}^{n} \theta_i x^i
\end{equation}

% Explain the equation
Where:
\begin{itemize}
    \item $y$ represents the output variable (e.g., strength, temperature)
    \item $x$ is the input variable (e.g., time, load)
    \item $\theta = [\theta_1, \theta_2, \ldots, \theta_n]$ is the parameter vector
    \item $n$ is the model order, determined by cross-validation
\end{itemize}

% Include additional mathematical derivations and proofs here
% Show your work step-by-step for transparency and reproducibility

%------------------------------------------------------------------------------
% SECTION 3: RESULTS
%------------------------------------------------------------------------------
\section{Results}
\label{sec:results}
\hypertarget{results}{}            % External link: report.pdf#results

% ← REPLACE with your results introduction
The experimental data shows excellent agreement with the theoretical thermal 
response model. Figure~\ref{fig:temperature_time} presents the measured 
temperature evolution compared to the model prediction.

%..............................................................................
% SHARED FIGURE: Using the reusable figure command
%..............................................................................
% This figure uses \figureTempData defined in template_shared_elements.tex
% The SAME figure command is used in the presentation slides
\begin{figure}[htbp]
    \centering
    \figureTempData   % ← Reusable command from template_shared_elements.tex
    \caption{Experimental temperature measurements (blue circles) compared with 
             theoretical model prediction (red dashed line). The exponential heating 
             model $\eqThermalResponse$ shows excellent agreement with experimental 
             data ($R^2 = 0.98$), validating the first-order thermal response 
             assumption. Extracted parameters: \thermalConstant, \ambientTemp.}
    \label{fig:temperature_time}
\end{figure}

%..............................................................................
% SUBSECTION: Quantitative Findings
%..............................................................................
\subsection{Quantitative Findings}
\label{subsec:quantitative}
\hypertarget{quantitative}{}       % External link: report.pdf#quantitative

% ← REPLACE with your quantitative results
The thermal time constant \thermalConstant~was extracted from least-squares 
fitting of the model equation $\eqThermalResponse$ to experimental data.

This section can be extensive with:
\begin{itemize}
    \item Multiple sub-analyses
    \item Statistical significance tests (p-values, confidence intervals)
    \item Error analysis and uncertainty quantification
    \item Sensitivity studies showing robustness of results
\end{itemize}

%..............................................................................
% SUBSECTION: Qualitative Observations
%..............................................................................
\subsection{Qualitative Observations}
\label{subsec:qualitative}
\hypertarget{qualitative}{}        % External link: report.pdf#qualitative

% ← REPLACE with your qualitative findings
Discuss qualitative findings, patterns observed, and interpretations that 
require extended discussion:

\begin{itemize}
    \item Visual patterns in data or images
    \item Unexpected behaviors or anomalies
    \item Trends that emerged during experimentation
    \item Observations that inform future work
\end{itemize}

%------------------------------------------------------------------------------
% SECTION 4: DISCUSSION
%------------------------------------------------------------------------------
\section{Discussion}
\label{sec:discussion}
\hypertarget{discussion}{}         % External link: report.pdf#discussion

% ← REPLACE with your in-depth discussion
Provide in-depth discussion of results, including:

\begin{itemize}
    \item \textbf{Interpretation in context:} How do your results compare to 
          existing literature? Do they support or contradict previous findings?
    
    \item \textbf{Limitations and uncertainties:} What are the sources of error? 
          What assumptions did you make? How might these affect conclusions?
    
    \item \textbf{Implications:} What do your results mean for theory and practice? 
          How might they be applied in real-world scenarios?
    
    \item \textbf{Unexpected findings:} Did anything surprise you? Why might this 
          have occurred? What does it suggest for future work?
\end{itemize}

%------------------------------------------------------------------------------
% SECTION 5: CONCLUSION
%------------------------------------------------------------------------------
\section{Conclusion}
\label{sec:conclusion}
\hypertarget{conclusion}{}         % External link: report.pdf#conclusion

% ← REPLACE with your conclusion
Summarize the key findings and their significance. Discuss future work 
directions and broader impacts.

% Structure your conclusion to:
% 1. Restate the problem and objectives
% 2. Summarize main findings (bullet points work well)
% 3. State the significance/contribution
% 4. Identify limitations
% 5. Suggest future directions

%..............................................................................
% SUBSECTION: Key Contributions
%..............................................................................
\subsection{Key Contributions}
\label{subsec:contributions}
\hypertarget{contributions}{}      % External link: report.pdf#contributions

% ← REPLACE with your key contributions
List the main contributions of this work:

\begin{enumerate}
    \item \textbf{Contribution 1:} Brief description of first major contribution
    \item \textbf{Contribution 2:} Brief description of second major contribution
    \item \textbf{Contribution 3:} Brief description of third major contribution
\end{enumerate}

%..............................................................................
% SUBSECTION: Future Work
%..............................................................................
\subsection{Future Work}
\label{subsec:future}
\hypertarget{future}{}             % External link: report.pdf#future

% ← REPLACE with your future work directions
Outline potential directions for future research and development:

\begin{itemize}
    \item Extension to other materials/conditions
    \item Refinement of methodology or model
    \item Investigation of observed anomalies
    \item Application to real-world problems
\end{itemize}

%------------------------------------------------------------------------------
% REPORT-ONLY SECTIONS (Not included in presentation)
%------------------------------------------------------------------------------

% Acknowledgments section (appears only in report, not in presentation)
\reportonly{
\section*{Acknowledgments}
% ← REPLACE with your acknowledgments
Acknowledge funding sources, collaborators, advisors, and others who contributed 
to the work. Include grant numbers if applicable.

% EXAMPLE:
% This work was supported by Grant No. ABC-123 from XYZ Foundation. 
% The authors thank Dr. Jane Smith for helpful discussions.
}

% Bibliography section (appears only in report, not in presentation)
\reportonly{
\bibliographystyle{plain}
% Uncomment the line below when you have a references.bib file
% \bibliography{references}

% To use bibliography:
% 1. Create a file called references.bib
% 2. Add your references in BibTeX format
% 3. Cite them in the text with \cite{key}
% 4. Uncomment the \bibliography line above
% 5. Compile with: pdflatex → bibtex → pdflatex → pdflatex
}

%==============================================================================
% TIPS FOR CUSTOMIZATION
%==============================================================================
%
% ADDING NEW SECTIONS:
%   Always include both \label{} and \hypertarget{} for new sections:
%   
%   \section{My New Section}
%   \label{sec:mynewsection}
%   \hypertarget{mynewsection}{}
%
% ADDING FIGURES:
%   Use template_shared_elements.tex for figures that appear in both formats.
%   For report-only figures:
%   
%   \begin{figure}[htbp]
%       \centering
%       \includegraphics[width=0.7\textwidth]{figures/myfigure.png}
%       \caption{Description of the figure}
%       \label{fig:myfigure}
%   \end{figure}
%
% CROSS-REFERENCING:
%   Reference figures: Figure~\ref{fig:myfigure}
%   Reference equations: Equation~\ref{eq:main}
%   Reference sections: Section~\ref{sec:introduction}
%
% REPORT-ONLY CONTENT:
%   Use \reportonly{} for content that should only appear in the report:
%   \reportonly{This detailed explanation is too long for slides.}
%
%==============================================================================
